%%&program=xelatex
%&encoding=UTF-8 Unicode
% SVN keywords
% $Author: bernardo $
% $Date: 2014-07-24 19:33:49 +0100 (Thu, 24 Jul 2014) $
% $Revision: 6559 $
% $U_aRL: http://metis.ipfn.ist.utl.pt/svn/cdaq/Users/Bernardo/Aulas/LFEB/teXfiles/Planck/Planck.tex $
\documentclass[a4paper,12pt]{article}  % Comments after  % are ignored
%\usepackage{hyperref}                 % For creating hyperlinks in cross references
%
\usepackage{ifxetex}% for XELATEX, or PDFlatex
\usepackage{ifplatform} 
%
\ifxetex
	\usepackage{polyglossia} \setmainlanguage{portuges}
	\usepackage{fontspec}
	\ifwindows
		\setmainfont[Ligatures=TeX]{Garamond}
		\setsansfont[Ligatures=TeX]{Gill Sans MT}
		\setmonofont{Consolas}
%		\setmonofont[Scale=MatchLowercase]{Courier}
		\setmonofont[Scale=0.95]{Courier}
	\fi
	\iflinux
		\setmainfont[Ligatures=TeX]{Linux Libertine O}
		\setsansfont[Ligatures=TeX,Scale=MatchLowercase]{Linux Biolinum}
		\setmonofont[Scale=MatchLowercase]{Courier}
	\fi
	\ifmacosx
	% add settings
	% Use xelatex -no-shell ...
		\setmainfont[Ligatures=TeX]{Garamond}
		\setsansfont[Ligatures=TeX]{Helvetica}
		\setmonofont{Consolas}
	\fi
	\usepackage{xcolor,graphicx} 
\else
	\usepackage[portuguese]{babel}
	%\usepackage[latin1]{inputenc}
	\usepackage[utf8]{inputenc}
	\usepackage[T1]{fontenc}
	\usepackage{graphics}                 % Packages to allow inclusion of graphics
	\usepackage{color}                    % For creating coloured text and background
\fi

\usepackage{enumitem}
\setlist{nolistsep}

\usepackage{amsmath,amssymb,amsfonts} % Typical maths resource packages 
\usepackage[retainorgcmds]{IEEEtrantools} 
\usepackage{caption}

\usepackage{tikz}
\usetikzlibrary{calc,arrows,decorations.pathmorphing,intersections}

\usepackage[font={small,sf},labelfont={bf},labelsep=endash]{caption}
\usepackage{sansmath}

\def\width{18}
\def\hauteur{11}

\oddsidemargin 0cm
\evensidemargin 0cm

\pagestyle{myheadings}         % Option to put page headers
                               % Needed \documentclass[a4paper,twoside]{article}

\markboth{{MEFT}}
{{\small\it \protect\input{../../LIFE.txt}}}

\addtolength{\hoffset}{-0.5cm}
\addtolength{\textwidth}{2.5cm}
\addtolength{\topmargin}{-1.5cm}
\addtolength{\textheight}{3cm}

%\textwidth 15.5cm
%\topmargin -1.5cm
\setlength{\parindent}{0pt}
\setlength{\parskip}{1ex  plus  0.5ex  minus  0.2ex}
%\parindent 0.5cm
%\textheight 25cm
%\parskip 1mm


% Math macros
\newcommand{\ud}{\,\mathrm{d}} 
\newcommand{\HRule}{\rule{\linewidth}{0.5mm}}

\author{Prof. Bernardo B. Carvalho} 

%%%%, Bernardo Brotas Carvalho\\bernardo.carvalho@tecnico.ulisboa.pt} 
\date{ Outubro 2015} 

\begin{document} 
{  \sf  Relatório da Experiência de Thomson} %[0.4cm] % \bfseries 

\input{../../Nomes.txt}


\section{\sf Trabalho preparatório a realizar ANTES da sessão de Laboratório:}
\begin{enumerate}
\item Descreva quais os objectivos do trabalho que irá realizar na sessão de laboratório. 
\item Desenhe um diagrama dos campos eléctricos, magnéticos, da velocidade dos electrões e forças aplicadas nas diferentes zonas do TRC, para a deflexão magnética e deflexão magnética e eléctrica.
\item Escolha os 5 pares de coordenadas, $(y,\, \pm z)$, na grelha do tubo TRC que irá utilizar nos ensaios de deflexão magnética, de modo a obter os maiores valores  de $R$ possíveis. Preencha as 3 primeiras colunas da Secção \ref{sec:dados}.
\end{enumerate}

\subsection{\sf Objectivos do Trabalho}
\noindent\underline{\makebox[\textwidth][r]{~}} \\
\noindent\underline{\makebox[\textwidth][r]{~}} \\
\noindent\underline{\makebox[\textwidth][r]{~}} \\
\noindent\underline{\makebox[\textwidth][r]{~}} \\
\noindent\underline{\makebox[\textwidth][r]{~}} \\
%\noindent\underline{\makebox[\textwidth][r]{~}} \\

\subsubsection{\sf Equações }
Escreva no seguinte quadro todas as equações necessárias para calcular as grandezas, bem como as suas incertezas e a lengenda de símbolos.  Numere as equações para futura referência. Indique nas tabelas qual a equação que utiliza para os cálculos.

%\begin{center}
\framebox[18cm]{\rule{0pt}{6.5cm}}
%\end{center}


\section{\sf Relatório}
\subsection{\sf DETERMINAÇÃO DE $q/m$ POR  DEFLEXÃO MAGNÉTICA}
\subsubsection{\sf Montagem Experimental}
Desenhe um diagrama da experiência. Inclua uma lista e legenda dos instrumentos e repectiva resolução e incerteza.
\begin{center}
\framebox[18cm]{\rule{0pt}{10cm}}
\end{center}



%\newpage

%\begin{figure}[!btp]
%     \makebox[\textwidth]{\framebox[18cm]{\rule{0pt}{7cm}}}
%     \caption{Montagem Experimental\label{montag}}
%\end{figure}

\subsubsection{\sf Medidas Experimentais e Cálculos Intermédios } \label{sec:dados}
Preencha as seguintes tabelas indicando apenas os algarismos significativos. Terá que verificar as contas com auxílio da calculadora, para um dos ensaios e na presença do docente. Indique as unidades de cada coluna, utilizando (sub)múltiplos mais adequados para o máximo de clareza nas tabelas.
%Poderá em alternativa utilizar folhas de cálculo, com o mesmo formato (apresentando-as em anexo) mas terá de preencher as colunas 1, 2, 3, 4 das tabelas seguintes e as colunas 1 e 6 das secção \ref{sec:calc}. 
%\newpage

\begin{center}
	\noindent	$U_a =$ \underline{\makebox[1.5cm][r]{~}} [~~~] ,  $\quad \delta U_a=$	\underline{\makebox[1cm][r]{~}} [~~~], $\quad \delta_y=$ \underline{\makebox[1cm][r]{~}} [mm],  $\quad \delta_z=$ \underline{\makebox[1cm][r]{~}} [mm] \\

	\renewcommand{\arraystretch}{0.75}
	
%\bigskip

	\begin{tabular}{|c|c|c|c|c|c|c|c|}
	\hline
	y [cm]  & $z_+ / z_- $ [cm]  & $R$ [~~] & $\delta R$ [~~] & $I_+$ [~~] & $I_-$ [~~~] & $\overline{I}$   {\tiny $ = \frac{| I_+ + I_-|}{2}$ [~~] } & {\tiny $\delta I = \frac{| I_+ - I_-|}{2}$ [~~] } \\
	\hline
	 &  &  &  & &  &  & \\ \cline{5-8}
	 &  &  &  & &  & & \\ \cline{5-8}
	 &  &  &  & &  & & \\ \cline{5-8}
	 \hline
	 &  &  &  & &  & & \\ \cline{5-8}
	 &  &  &  & &  & & \\ \cline{5-8}
	 &  &  &  & &  & & \\ \cline{5-8}
	 \hline
	 &  &  &  & &  & & \\ \cline{5-8}
	 &  &  &  & &  & & \\ \cline{5-8}
	 &  &  &  & &  & & \\ \cline{5-8}
	 \hline
	 &  &  &  & &  & & \\ \cline{5-8}
	 &  &  &  & &  & & \\ \cline{5-8}
	 &  &  &  & &  & & \\ \cline{5-8}
	 \hline
	 &  &  &  & &  & & \\ \cline{5-8}
	 &  &  &  & &  & & \\ \cline{5-8}
	 &  &  &  & &  & & \\ \cline{5-8}
	 \hline
 	\end{tabular}

	\bigskip
		
	\noindent	$U_a =$ \underline{\makebox[1.5cm][r]{~}}  $\pm$  	\underline{\makebox[1cm][r]{~}} V

	%\bigskip

	\begin{tabular}{|c|c|c|c|c|c|c|c|}
	\hline
	y [cm]  & $z_+ / z_- $ [cm]  & $R$ [~~] & $\delta R$ [~~] & $I_+$ [~~] & $I_-$ [~~] & $\overline{I}$   {\tiny $ = \frac{| I_+ + I_-|}{2}$ [~~] } & {\tiny $\delta I = \frac{| I_+ - I_-|}{2}$ [~~~] } \\
	\hline
	 &  &  &  & &  &  & \\ \cline{5-8}
	 &  &  &  & &  & & \\ \cline{5-8}
	 &  &  &  & &  & & \\ \cline{5-8}
	 \hline
	 &  &  &  & &  & & \\ \cline{5-8}
	 &  &  &  & &  & & \\ \cline{5-8}
	 &  &  &  & &  & & \\ \cline{5-8}
	 \hline
	 &  &  &  & &  & & \\ \cline{5-8}
	 &  &  &  & &  & & \\ \cline{5-8}
	 &  &  &  & &  & & \\ \cline{5-8}
	 \hline
	 &  &  &  & &  & & \\ \cline{5-8}
	 &  &  &  & &  & & \\ \cline{5-8}
	 &  &  &  & &  & & \\ \cline{5-8}
	 \hline
	 &  &  &  & &  & & \\ \cline{5-8}
	 &  &  &  & &  & & \\ \cline{5-8}
	 &  &  &  & &  & & \\ \cline{5-8}
	 \hline
 	\end{tabular}

	\bigskip

	\noindent	$U_a =$ \underline{\makebox[1.5cm][r]{~}}  $\pm$  	\underline{\makebox[1cm][r]{~}}V

	%\bigskip

	\begin{tabular}{|c|c|c|c|c|c|c|c|}
	\hline
	y [cm]  & $z_+ / z_-$ [cm]  & $R$ [~~~] & $\delta R$ [~~~] & $I_+$ [~~] & $I_-$ [~~] & $\overline{I}$   {\tiny $ = \frac{| I_+ + I_-|}{2}$ [~~] } & {\tiny $\delta \overline{I} = \frac{| I_+ - I_-|}{2}$ [~~] } \\
	\hline
	 &  &  &  & &  &  & \\ \cline{5-8}
	 &  &  &  & &  & & \\ \cline{5-8}
	 &  &  &  & &  & & \\ \cline{5-8}
	 \hline
	 &  &  &  & &  & & \\ \cline{5-8}
	 &  &  &  & &  & & \\ \cline{5-8}
	 &  &  &  & &  & & \\ \cline{5-8}
	 \hline
	 &  &  &  & &  & & \\ \cline{5-8}
	 &  &  &  & &  & & \\ \cline{5-8}
	 &  &  &  & &  & & \\ \cline{5-8}
	 \hline
	 &  &  &  & &  & & \\ \cline{5-8}
	 &  &  &  & &  & & \\ \cline{5-8}
	 &  &  &  & &  & & \\ \cline{5-8}
	 \hline
	 &  &  &  & &  & & \\ \cline{5-8}
	 &  &  &  & &  & & \\ \cline{5-8}
	 &  &  &  & &  & & \\ \cline{5-8}
	 \hline
 	\end{tabular}
\end{center}

%\begin{table}[!hbp]
%\begin{center}
%	\centering
%	\noindent	$U_a =$ \underline{\makebox[1.5cm][r]{~}} $\pm$  	\underline{\makebox[1cm][r]{~}} V \\

%	\caption{Dados da Experiência de Deflexão Magnética} 
%	\label{tab:Dados}
%\end{table}


\subsubsection{\sf Cálculos de $q/m$}
\label{sec:calc}
%\vspace{1cm}
\begin{center}
	%\bigskip
	\noindent	$R =$ \underline{\makebox[1.5cm][r]{~}}  $\pm$  	\underline{\makebox[1cm][r]{~}}  [~~~]
%	\bigskip
	\begin{tabular}{|c|c|c|c|c|c|c|}
	\hline %\cline{2-7}
	  $U_a$ [~~~] & $\overline{I}$ [~~] &  $B$ [~~~] & $\delta B$  [~~~] & $q/m$ [$10^{11}$C/kg] & $\delta q/m$ [$10^{11}$C/kg] & $\overline{q/m}$ [$10^{11}$C/kg]\\ \hline %\cline{2-7}
	% &$\qquad \pm \quad$&&&&&  \\ \cline{2-6}
	 &$\qquad \pm \quad$&&&&& \\ \cline{2-6}
	$\qquad \pm \quad$ &$\qquad \pm \quad$&&&&& $\qquad \pm \quad$   \\ \cline{2-6}
	 %&$\qquad \pm \quad$&&&&& \\ \cline{2-6}
	 &$\qquad \pm \quad$&&&&& \\ \hline 
 	\end{tabular}

 	%\bigskip
	\noindent	$R =$ \underline{\makebox[1.5cm][r]{~}}  $\pm$  	\underline{\makebox[1cm][r]{~}}  [~~~]
	\begin{tabular}{|c|c|c|c|c|c|c|}  
	\hline %\cline{2-7}
	  $U_a$ [~~~] & $\overline{I}$ [~~] &  $B$ [~~~] & $\delta B$  [~~~] & $q/m$ [$10^{11}$C/kg] & $\delta q/m$ [$10^{11}$C/kg] & $\overline{q/m}$ [$10^{11}$C/kg]\\ \hline 
	 $\qquad \pm \quad$ &$\qquad \pm \quad$&&&&& \\ \cline{2-6}
	 $\qquad \pm \quad$ &$\qquad \pm \quad$&&&&& $\qquad \pm \quad$ \\ \cline{2-6}
	 $\qquad \pm \quad$ &$\qquad \pm \quad$&&&&& \\
	 \hline 
	\end{tabular}

	\noindent	$R =$ \underline{\makebox[1.5cm][r]{~}}  $\pm$  	\underline{\makebox[1cm][r]{~}}  [~~~]
	\begin{tabular}{|c|c|c|c|c|c|c|}  
	\hline %\cline{2-7}
	  $U_a$ [~~~] & $\overline{I}$ [~~] &  $B$ [~~~] & $\delta B$  [~~~] & $q/m$ [$10^{11}$C/kg] & $\delta q/m$ [$10^{11}$C/kg] & $\overline{q/m}$ [$10^{11}$C/kg]\\ \hline 
	 $\qquad \pm \quad$ &$\qquad \pm \quad$&&&&& \\ \cline{2-6}
	 $\qquad \pm \quad$ &$\qquad \pm \quad$&&&&& $\qquad \pm \quad$ \\ \cline{2-6}
	 $\qquad \pm \quad$ &$\qquad \pm \quad$&&&&& \\
	 \hline 
	\end{tabular}

 \noindent	$R =$ \underline{\makebox[1.5cm][r]{~}}  $\pm$  	\underline{\makebox[1cm][r]{~}}  [~~~]
	\begin{tabular}{|c|c|c|c|c|c|c|}  
	\hline %\cline{2-7}
	  $U_a$ [~~~] & $\overline{I}$ [~~] &  $B$ [~~~] & $\delta B$  [~~~] & $q/m$ [$10^{11}$C/kg] & $\delta q/m$ [$10^{11}$C/kg] & $\overline{q/m}$ [$10^{11}$C/kg]\\ \hline 
	 $\qquad \pm \quad$ &$\qquad \pm \quad$&&&&& \\ \cline{2-6}
	 $\qquad \pm \quad$ &$\qquad \pm \quad$&&&&& $\qquad \pm \quad$ \\ \cline{2-6}
	 $\qquad \pm \quad$ &$\qquad \pm \quad$&&&&& \\
	 \hline 
	\end{tabular}

\noindent	$R =$ \underline{\makebox[1.5cm][r]{~}}  $\pm$  	\underline{\makebox[1cm][r]{~}}  [~~~]
	\begin{tabular}{|c|c|c|c|c|c|c|}  
	\hline %\cline{2-7}
	  $U_a$ [~~~] & $\overline{I}$ [~~] &  $B$ [~~~] & $\delta B$  [~~~] & $q/m$ [$10^{11}$C/kg] & $\delta q/m$ [$10^{11}$C/kg] & $\overline{q/m}$ [$10^{11}$C/kg]\\ \hline 
	 $\qquad \pm \quad$ &$\qquad \pm \quad$&&&&& \\ \cline{2-6}
	 $\qquad \pm \quad$ &$\qquad \pm \quad$&&&&& $\qquad \pm \quad$ \\ \cline{2-6}
	 $\qquad \pm \quad$ &$\qquad \pm \quad$&&&&& \\
	 \hline 
	\end{tabular}
\end{center}

Incertezas relativas parciais

	\begin{small}
	\begin{tabular}{|c|c|c|c|c|c|c|}  
	\hline %\cline{2-7}
	  $\delta_{(U_a)}  q/m$ [~~~] &  $ \delta_{(U_a)} q/m$ [\%] & $\delta_{(R)}  q/m$ [~~~] & $\delta_{(R)}  q/m$ [\%] & 
	  $\delta_{(\overline{I})}  q/m $ [~~~] & $\delta_{(\overline{I})}  q/m $ [\%] & $\delta  q/m$ [$10^{11}$C/kg]\\ \hline 
 %	\\ \hline
	 &&&&&& \\ 
	 &&&&&& \\ \hline
	\end{tabular}
	 \end{small}

\subsubsection{\sf Resultados Finais. Explique os critérios que utilizou para obter as incertezas.}
\noindent  $q/m_{(B)} =$~(\underline{\makebox[1.5cm][r]{~}}$\pm$\underline{\makebox[1cm][r]{~}})$\times 10^{11}\,\,$C/kg  \\  

\noindent  Desvio à Exactidão $=$~\underline{\makebox[1cm][r]{~}}\%, 
Incerteza relativa $=$~\underline{\makebox[1cm][r]{~}}\% 

\noindent\underline{\makebox[\textwidth][r]{~}} \\
\noindent\underline{\makebox[\textwidth][r]{~}} \\
\noindent\underline{\makebox[\textwidth][r]{~}} \\
\noindent\underline{\makebox[\textwidth][r]{~}} \\
\noindent\underline{\makebox[\textwidth][r]{~}} \\

%\newpage

\subsection{\sf DETERMINAÇÃO DE $q/m$ POR DEFLEXÃO\\ MAGNÉTICA E ELÉTRICA QUASE COMPENSADAS }

\subsubsection{\sf Dados Experimentais e Cálculos}

\noindent  Distância entre placas  $d=$~\underline{\makebox[1cm][r]{~}}[~~~] 

%\begin{itemize}
%\item Preencha as tabelas seguintes:
%\end{itemize}
\begin{table}[!hbp]
\begin{small}
	\centering
%	\noindent	$U_a =$ \underline{\makebox[1.5cm][r]{~}} $V$ \\ %$\pm$ \underline{\makebox[1cm][r]{~}} 

	\noindent	$U_a =$ \underline{\makebox[1.5cm][r]{~}} $\pm$ \underline{\makebox[1cm][r]{~}} V \\
	\begin{tabular}{|c|c|c|c|c|c|c|c|}
	\hline
	 $I_{max}$ [~~~] & $I_{min}$ [~~~] & $\overline{I}$ [~~~]	& $\delta I $ [~~~] & $B$ [~~~] & $\delta B$ [~~~] & $q/m$ [$10^{11}$C/kg] & $\delta q/m$ [$10^{11}$C/kg] \\
	\hline
	 &  &  & &  &  & & \\
	 \hline
 	\end{tabular}\\[10pt]
	%= \frac{| I_{max} - I_{min}|}{2}
	\bigskip
	
	\noindent	$U_a =$ \underline{\makebox[1.5cm][r]{~}} $\pm$ \underline{\makebox[1cm][r]{~}} V \\
	\begin{tabular}{|c|c|c|c|c|c|c|c|}
	\hline
	 $I_{max}$ [~~~] & $I_{min}$ [~~~] & $\overline{I}$ [~~~]	& $\delta I $ [~~~] & $B$ [~~~] & $\delta B$ [~~~] & $q/m$ [$10^{11}$C/kg] & $\delta q/m$ [$10^{11}$C/kg] \\
	\hline
	 &  &  & &  &  & & \\
	 \hline
 	\end{tabular}\\[10pt]

	\bigskip
	
	\noindent	$U_a =$ \underline{\makebox[1.5cm][r]{~}} $\pm$ \underline{\makebox[1cm][r]{~}} V \\
	\begin{tabular}{|c|c|c|c|c|c|c|c|}
	\hline
	 $I_{max}$ [~~~] & $I_{min}$ [~~~] & $\overline{I}$ [~~~]	& $\delta \overline{I} $ [~~~] & $B$ [~~~] & $\delta B$ [~~~] & $q/m$ [$10^{11}$C/kg] & $\delta q/m$ [$10^{11}$C/kg] \\
	\hline
	 &  &  & &  &  & & \\
	 \hline
 	\end{tabular}
	
%\	\caption{Dados da Experiência 2} 
	\label{tab:Dados2}
\end{small}
\end{table}

%= \frac{| I_{max} - I_{min}|}{2}

\subsubsection{\sf Resultados}
\noindent  $q/m_{(B,E)} =$~(\underline{\makebox[1.5cm][r]{~}}$\pm$\underline{\makebox[1cm][r]{~}})$\times 10^{11}\,$ C/kg  \\  

\noindent  Desvio à Exatidão $=$~\underline{\makebox[1cm][r]{~}}\%, 
Incerteza relativa $=$~\underline{\makebox[1cm][r]{~}}\% 

\subsection{\sf Trajetória não compensada}
Aumente agora o campo $B$ (sempre com $I\leq 3$ A) de forma a visualizar uma trajetória claramente não compensada.  Faça um esboço da curva observada, indicando os vetores das forças em jogo (com uma estimativa do seu valor em [N]), bem como as condições experimentais. Comente a figura obtida.
\begin{center}
\framebox[18cm]{\rule{0pt}{6.5cm}}
\end{center}

\subsection{\sf Análise e comparação dos dois métodos. Conclusões e Comentários Finais}
\noindent\underline{\makebox[\textwidth][r]{~}} \\
\noindent\underline{\makebox[\textwidth][r]{~}} \\
\noindent\underline{\makebox[\textwidth][r]{~}} \\
\noindent\underline{\makebox[\textwidth][r]{~}} \\
\noindent\underline{\makebox[\textwidth][r]{~}} \\
\noindent\underline{\makebox[\textwidth][r]{~}} \\
\noindent\underline{\makebox[\textwidth][r]{~}} \\
\noindent\underline{\makebox[\textwidth][r]{~}} \\
\noindent\underline{\makebox[\textwidth][r]{~}} \\
\noindent\underline{\makebox[\textwidth][r]{~}} \\
\noindent\underline{\makebox[\textwidth][r]{~}} \\
\noindent\underline{\makebox[\textwidth][r]{~}} \\
\noindent\underline{\makebox[\textwidth][r]{~}} \\
\noindent\underline{\makebox[\textwidth][r]{~}} \\
\noindent\underline{\makebox[\textwidth][r]{~}} \\
\noindent\underline{\makebox[\textwidth][r]{~}} \\
\noindent\underline{\makebox[\textwidth][r]{~}} \\
\noindent\underline{\makebox[\textwidth][r]{~}} \\
\noindent\underline{\makebox[16.2cm][r]{~}} \\
%\begin{center}
%     \makebox[\textwidth]{\framebox[18cm]{\rule{0pt}{5cm}}}
%     \caption{Montagem Experimental\label{montag}}
%\end{center}



%\newpage
%\section*{\sf Apêndice}



\end{document} 