%%&program=xelatex
%&encoding=UTF-8 Unicode
\documentclass[a4paper,12pt]{article}  % Comments after  % are ignored
%\usepackage{hyperref}                 % For creating hyperlinks in cross references
%
\usepackage{ifxetex}% for XELATEX, or PDFlatex
\usepackage{ifplatform} 
%
\ifxetex
	\usepackage{polyglossia} \setmainlanguage{portuges}
	\usepackage{fontspec}
	\ifwindows
		\setmainfont[Ligatures=TeX]{Garamond}
		\setsansfont[Ligatures=TeX]{Arial}
		\setmonofont{Consolas}
%		\setmonofont[Scale=MatchLowercase]{Courier}
		\setmonofont[Scale=0.95]{Courier}
	\fi
	\iflinux
		\setmainfont[Ligatures=TeX]{Linux Libertine O}
		\setsansfont[Ligatures=TeX,Scale=MatchLowercase]{Linux Biolinum}
		\setmonofont[Scale=MatchLowercase]{Courier}
	\fi
	\ifmacosx
	% add settings
	% Use xelatex -no-shell ...
		\setmainfont[Ligatures=TeX]{Garamond}
		\setsansfont[Ligatures=TeX]{Helvetica}
		\setmonofont{Consolas}
	\fi
	\usepackage{xcolor,graphicx} 
\else
	\usepackage[portuguese]{babel}
	%\usepackage[latin1]{inputenc}
	\usepackage[utf8]{inputenc}
	\usepackage[T1]{fontenc}
	\usepackage{graphics}                 % Packages to allow inclusion of graphics
	\usepackage{color}                    % For creating coloured text and background
\fi

\usepackage{enumitem}
\setlist{nolistsep}

\usepackage{amsmath,amssymb,amsfonts} % Typical maths resource packages 
%\usepackage[retainorgcmds]{IEEEtrantools} 
\usepackage{caption}

\usepackage{tikz} %install pgf
\usetikzlibrary{calc,arrows,decorations.pathmorphing,intersections}

\usepackage[font={small,sf},labelfont={bf},labelsep=endash]{caption}
\usepackage{sansmath}

\def\width{18}
\def\hauteur{11}

\oddsidemargin 0cm
\evensidemargin 0cm

\pagestyle{myheadings}         % Option to put page headers
                               % Needed \documentclass[a4paper,twoside]{article}

\markboth{{MEFT}}
{{\small\it \protect\input{../../LIFE.txt}}}

\addtolength{\hoffset}{-0.5cm}
\addtolength{\textwidth}{2.5cm}
\addtolength{\topmargin}{-1.5cm}
\addtolength{\textheight}{3cm}

%\textwidth 15.5cm
%\topmargin -1.5cm
\setlength{\parindent}{0pt}
\setlength{\parskip}{1ex  plus  0.5ex  minus  0.2ex}
%\parindent 0.5cm
%\textheight 25cm
%\parskip 1mm


% Math macros
\newcommand{\ud}{\,\mathrm{d}} 
\newcommand{\HRule}{\rule{\linewidth}{0.5mm}}

\author{Prof. Bernardo B. Carvalho} 

%%%%, Bernardo Brotas Carvalho\\bernardo.carvalho@tecnico.ulisboa.pt} 
\date{ Setembro 2017} 

\begin{document} 
{  \sf  Relatório da Experiência de Propagação de Ondas Mecânicas} %[0.4cm] % \bfseries 

\input{../../Nomes.txt}


\section{\sf Trabalho preparatório a realizar  ANTES da sessão de Laboratório:}
\begin{enumerate}
\item Descreva por palavras suas quais os objectivos do trabalho que irá realizar.

%\subsubsection{\sf Questões a responder ANTES da sessão de Laboratório:}
%\begin{enumerate}
% (uma folha A4). Indique as expressões que irá utilizar para obter as grandezas experimentais, bem como as expressões para calcular as incertezas. Inclua esta parte também no Relatório. Este irá constituir o ÚNICO meio de consulta na Prova Individual.
%\item A luz emitida pelo díodo vermelho tem um comprimento de onda de $\lambda \approx 600 \, nm$. Calcule $f_{luz}$ e compare com a frequência da modulação da amplitude.
%\item Explique porque razão nesta experiência a luz utilizada  tem de ter a intensidade variável.
%, tal como na experiência da Velocidade do Som.
%\item Uma outra montagem desenvolvida no IST utiliza um díodo laser de longo alcance. No entanto o máximo que se consegue modular a amplitude é, $f_{mod}=1\, MHz$. Calcule a distância
%dos espelhos necessária para se obter o efeito de oposição de fase entre a amplitude dos sinais emitido e o recebido.
\end{enumerate}

\noindent\underline{\makebox[\textwidth][r]{~}} \\
\noindent\underline{\makebox[\textwidth][r]{~}} \\
\noindent\underline{\makebox[\textwidth][r]{~}} \\
\noindent\underline{\makebox[\textwidth][r]{~}} \\
\noindent\underline{\makebox[\textwidth][r]{~}} \\
\noindent\underline{\makebox[\textwidth][r]{~}} \\
\noindent\underline{\makebox[\textwidth][r]{~}} \\
\noindent\underline{\makebox[\textwidth][r]{~}} \\
\noindent\underline{\makebox[\textwidth][r]{~}} \\
\noindent\underline{\makebox[\textwidth][r]{~}} \\


% (uma folha A4). 
%Indique as expressões que irá utilizar para obter as grandezas experimentais, bem como as expressões para calcular as incertezas. Inclua esta parte também no Relatório. Este irá constituir o ÚNICO meio de consulta na Prova Individual.



\subsubsection{\sf Equações }
Escreva no seguinte quadro todas as equações necessárias para calcular as grandezas bem com as suas incertezas.
\begin{center}
\framebox[15cm]{\rule{0pt}{6.5cm}}
\end{center}


\section{\sf Relatório}
\subsection{\sf Montagem Experimental}
Desenhe um diagrama da experiência, bem como um esboço das imagem que observa no osciloscópio. Inclua uma lista com a legenda de instrumentos.

\begin{center}
\framebox[18cm]{\rule{0pt}{7.5cm}}
\end{center}

\subsection{\sf Calibração dos Sensores P \& S com cilindros de latão}%\label{sec:dados}
%\subsubsection{\sf Dados Experimentais}\label{sec:dados}
Preencha as  tabelas indicando  apenas os algarismos significativos. 

Incerteza na medida do comprimento da amostra, $e_L=$ ~\underline{\makebox[1cm][r]{~}} mm% \%

\begin{center}
	%\centering
	\begin{tabular}{|l|c|c|c|c|c|c|}
	\hline
	 L (mm)  & & & & & &  \\
	\hline 
	 P-Tempo (ms) & $ \quad \quad \pm \quad $ &$ \quad \quad \pm \quad $ &$ \quad \quad \pm \quad $ &$ \quad \quad \pm \quad $ &$ \quad \quad \pm \quad $ &$ \quad \quad \pm \quad $ \\
\hline
     S-Tempo (ms) & $ \quad \quad \pm \quad $ &$ \quad \quad  \pm \quad $ &$ \quad  \quad \pm \quad $ &$ \quad  \quad \pm \quad $ &$ \quad \quad  \pm \quad $ &$ \quad  \quad \pm \quad $ \\
\hline
				\end{tabular}
\end{center}

Represente graficamente o tempo de propagação \emph{vs} comprimento e por Regressão Linear obtenha o melhor ajuste a uma recta.

Sensor P

Declive: $m=$ ~\underline{\makebox[2cm][r]{~}} s/m, Ordenada na origem: $\Delta T_0=$ ~\underline{\makebox[2cm][r]{~}} ms\\
(Este último valor pode constituir uma estimativa do erro sistemático da medida de intervalo de tempo para as ondas P.)

Sensor S

Declive: $m=$ ~\underline{\makebox[2cm][r]{~}} s/m, Ordenada na origem: $\Delta T_0=$ ~\underline{\makebox[2cm][r]{~}} ms\\
(Este último valor pode constituir uma estimativa do erro sistemático da medida de intervalo de tempo as ondas S.)


 %
\subsection{\sf Velocidade de propagação em meios sólidos}%\label{sec:dados}
%\subsubsection{\sf Dados Experimentais}\label{sec:dados}

%Poderá em alternativa utilizar folhas de cálculo, com o mesmo formato (apresentando-as em anexo) mas terá de peencher as colunas 2, 3, 5 e 6 da tabela seguintes e as colunas 6 e 7 das secção \ref{sec:calc}. 
Nota: Terá que verificar as contas com auxílio da calculadora, para um dos ensaios e na presença do docente.

%Execute as Medições e preencha a tabela seguinte :

%\noindent  Amostra 1$=$~\underline{\makebox[2cm][r]{~}}, 
%Dimensões: $L_x=$~\underline{\makebox[2cm][r]{~}}
%$\pm$ \underline{\makebox[1cm][r]{~}} m, 
%$L_y=$~\underline{\makebox[2cm][r]{~}}
%$\pm$ \underline{\makebox[1cm][r]{~}} m 

Dimensões e densidades:
\begin{center}
	%\centering
	\begin{tabular}{|l|c|c|c|c|c|c|c|}
	\hline
	 Amostra \# & Material  &  $L_x$ [m]  &   $L_y$ [m]  &  $L_z$ [m]  & $Vol$ [$m^3$]  & Massa [$kg$] & $\rho$ [$kg/m^3$] \\
	\hline \hline
	  1   &   & $ \quad  \quad \pm \quad $ &  $ \quad \quad  \pm \quad $ & $ \quad \quad  \pm \quad $ & $ \quad  \quad \pm \quad $ & $ \quad \quad  \pm \quad $ & $ \quad  \quad \pm \quad $ \\ \hline
	  2   &   & $ \quad  \quad \pm \quad $ &  $ \quad \quad  \pm \quad $ & $ \quad  \quad \pm \quad $ & $ \quad  \quad \pm \quad $ & $ \quad  \quad \pm \quad $ & $ \quad \quad  \pm \quad $ \\ \hline
	  3   &   & $ \quad  \quad \pm \quad $ &  $ \quad  \quad \pm \quad $ & $ \quad  \quad \pm \quad $ & $ \quad  \quad \pm \quad $ & $ \quad  \quad \pm \quad $ & $ \quad \quad  \pm \quad $ \\ \hline
	  4   &   & $ \quad  \quad \pm \quad $ &  $ \quad \quad  \pm \quad $ & $ \quad \quad  \pm \quad $ & $ \quad  \quad \pm \quad $ & $ \quad \quad  \pm \quad $ & $ \quad \quad  \pm \quad $ \\ \hline
				\end{tabular}
\end{center}

%$=$~\underline{\makebox[cm][r]{~}} m , 

Tempos e velocidades:
\begin{center}
	%\centering
	\begin{tabular}{|l|c|c|c|c|c|	}
	\hline
	  Amostra \# &   $t_x$ [ms]  &   $t_y$ [ms]  &   $v_x$ [m/s] & $v_y$ [m/s] &  
	  $c.a.$	\\
	\hline \hline
	  1 - Onda P   & $ \quad \quad \pm \quad $  & $ \quad \quad\pm \quad $ & $ \quad \quad\pm \quad $ & $ \quad\quad \pm \quad $ & $ \quad \quad\pm \quad $ \\ \hline
	  1 - Onda S   & $ \quad \quad \pm \quad $  & $ \quad \quad\pm \quad $ & $ \quad \quad\pm \quad $ & $ \quad\quad \pm \quad $ & $ \quad \quad\pm \quad $ \\ \hline
	  2 - Onda P   & $ \quad \quad\pm \quad $  & $ \quad \quad\pm \quad $ & $ \quad \quad\pm \quad $ & $ \quad\quad \pm \quad $ & $ \quad \quad\pm \quad $ \\ \hline
	  2 - Onda S   & $ \quad \quad\pm \quad $  & $ \quad \quad\pm \quad $ & $ \quad \quad\pm \quad $ & $ \quad\quad \pm \quad $ & $ \quad \quad\pm \quad $ \\ \hline
	  3 - Onda P   & $ \quad \quad\pm \quad $  & $ \quad \quad\pm \quad $ & $ \quad \quad\pm \quad $ & $ \quad\quad \pm \quad $ & $ \quad \quad\pm \quad $ \\ \hline
	  3 - Onda S   & $ \quad \quad\pm \quad $  & $ \quad \quad\pm \quad $ & $ \quad \quad\pm \quad $ & $ \quad\quad \pm \quad $ & $ \quad \quad\pm \quad $ \\ \hline
	  4 - Onda P   & $ \quad \quad\pm \quad $  & $ \quad \quad\pm \quad $ & $ \quad \quad\pm \quad $ & $ \quad\quad \pm \quad $ & $ \quad \quad\pm \quad $ \\ \hline
	  4 - Onda S   & $ \quad \quad\pm \quad $  & $ \quad \quad\pm \quad $ & $ \quad \quad\pm \quad $ & $ \quad\quad \pm \quad $ & $ \quad \quad\pm \quad $ \\ \hline
 	\end{tabular}
\end{center}

\subsubsection{\sf Cálculos de constantes elásticas}%\label{sec:dados}
Seleccione as amostras isótropas e cálcule a velocidade média e as constantes elásticas médias. 

\begin{center}
	%\centering
	\begin{tabular}{|l|c|c|c|c|c|c|}
	\hline
	  Material &   $v_P$ [m/s]  &   $\mu$ [GPa]  &  $v_S$ [m/s] & $K$ [GPa] & $v_P/v_S$  &  $\sigma$  	\\
	\hline \hline
	    & $ \quad \quad \pm \quad $ &  $ \quad \quad \pm \quad $ & $ \quad \quad \pm \quad $ & $ \quad \quad \pm \quad $ & $ \quad \quad  \pm \quad $ & $ \quad \quad \pm \quad $ \\ \hline
 	     & $ \quad \quad \pm \quad $ &  $ \quad \quad  \pm \quad $ & $ \quad \quad \pm \quad $ & $ \quad \quad \pm \quad $ & $ \quad \quad \pm \quad $ & $ \quad \quad \pm \quad $ \\ \hline
 	\end{tabular}
\end{center}

%\noindent Desvio à exactidão de $\overline{c_{ar}} =$~\underline{\makebox[1cm][r]{~}} \%, 
Incerteza relativa de $\mu=$ ~\underline{\makebox[1cm][r]{~}} \%

%\vspace{4cm}
\subsection{\sf Velocidade de propagação em meio liquido}%\label{sec:dados}


%\begin{center}
%	%\centering
%	\begin{tabular}{|l|c|c|c|c|c|c|c|}
%	\hline
%	 Amostra \# & Material   &  $L_z$ [m]  & $Vol$ [$m^3$]  & Massa [$kg$] & $\rho$ [$kg/m^3$] \\
%	\hline \hline
%	  3   &   & $ \quad \pm \quad $ &  $ \quad \pm \quad $ & $ \quad \pm \quad $ & $ \quad \pm \quad $ & $ \%quad \pm \quad $ & $ \quad \pm \quad $ \\ \hline
%	\end{tabular}
%\end{center}

Considere a densidade padrão da água.

Distância, tempos e velocidades:
\begin{center}
	%\centering
	\begin{tabular}{|l|c|c|c|}
	\hline
	  Água &   $L$ [mm] &   $t$ [ms]  &   $v$ [m/s] \\
	\hline \hline
	 Onda P   & $ \quad \quad \pm \quad $ &  $ \quad \quad \pm \quad $ & $ \quad \quad \pm \quad $  \\ \hline
	 Onda S   & $ \quad \quad \pm \quad $ &  $ \quad \quad \pm \quad $ & $ \quad \quad \pm \quad $  \\ \hline
 	\end{tabular}
\end{center}

\subsubsection{\sf Cálculos de constantes elásticas}%\label{sec:dados}

\begin{center}
	%\centering
	\begin{tabular}{|l|c|c|c|c|c|c|}
	\hline
	  Amostra &   $v_P$ [m/s]  &   $\mu$ [GPa]  &  $v_{S}$ [m/s] & $K$ [GPa]  	\\
	\hline \hline
	  Água   & $ \quad  \quad \pm \quad $ &  $ \quad  \quad \pm \quad $ & $ \quad  \quad \pm \quad $ & $ \quad \quad  \pm \quad $ \\ \hline
 	\end{tabular}
\end{center}
%\noindent Desvio à exactidão de $\overline{c_{ar}} =$~\underline{\makebox[1cm][r]{~}} \%, 
Incerteza relativa de $K=$ ~\underline{\makebox[1cm][r]{~}} \%



%\noindent  Desvio à Exatidão $=$~\underline{\makebox[1cm][r]{~}}(\%), 
%Incerteza relativa $=$~\underline{\makebox[1cm][r]{~}}($\%$) 

\subsection{\sf Análise, Conclusões e Comentários}
\noindent\underline{\makebox[\textwidth][r]{~}} \\
\noindent\underline{\makebox[\textwidth][r]{~}} \\
\noindent\underline{\makebox[\textwidth][r]{~}} \\
\noindent\underline{\makebox[\textwidth][r]{~}} \\
\noindent\underline{\makebox[\textwidth][r]{~}} \\
\noindent\underline{\makebox[\textwidth][r]{~}} \\
\noindent\underline{\makebox[\textwidth][r]{~}} \\
\noindent\underline{\makebox[\textwidth][r]{~}} \\
\noindent\underline{\makebox[\textwidth][r]{~}} \\
\noindent\underline{\makebox[\textwidth][r]{~}} \\
\noindent\underline{\makebox[\textwidth][r]{~}} \\
\noindent\underline{\makebox[\textwidth][r]{~}} \\
\noindent\underline{\makebox[\textwidth][r]{~}} \\
\noindent\underline{\makebox[\textwidth][r]{~}} \\
\noindent\underline{\makebox[\textwidth][r]{~}} \\




\end{document} 